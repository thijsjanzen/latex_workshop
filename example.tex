\documentclass{article}

\usepackage{graphicx} % Required for inserting images
\usepackage{natbib}

\title{LateX workshop}
\author{Thijs Janzen}
\date{February 2023}

\begin{document}

\maketitle

\section{Introduction}
Welcome to the \LaTeX Workshop. \\
We can \textit{emphasize} \textbf{important} words.
You can also do \textit{\textbf{both}}.

\section*{Math}
The frequency is given by $1 - p$. \\
Thus we find:
\begin{equation}
\label{eq:einstein}
    E = mc^2
\end{equation}

\section{Object} \label{sec:object}
\subsection{Figures}
\begin{figure}[p]
    \centering
    \includegraphics[width = \textwidth]{gopher.jpeg}
    \caption{This is an example of a Gopher.}
    \label{fig:gopher_picture}
\end{figure}

\subsection{Tables}

\begin{table}[h]
    \centering
    \begin{tabular}{|c|l|r|}
    \hline
        \textbf{Country}     & \textbf{Language} & \textbf{Food} \\
    \hline
        Greece      & Greek & Moussaka \\
        France      & French & Coq au Vin \\
        Netherlands & Dutch & Kroket \\
    \hline
    \end{tabular}
    \caption{Information about countries}
    \label{tab:country_table}
\end{table}


\section{Referencing}
\subsection{cross-referencing}
In section \ref{sec:object} we have shown in table \ref{tab:country_table} information about several countries, but see Figure \ref{fig:gopher_picture} and equation \ref{eq:einstein}.

\subsection{Citations}
More information about Gophers (see Figure \ref{fig:gopher_picture}) can also be found in \citep{gopher_paper}.

\section{Bibliography}
\bibliography{library.txt}
\bibliographystyle{apalike}

\end{document}
